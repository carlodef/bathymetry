\documentclass[9pt,xcolor=dvipsnames]{beamer}

\usepackage[T1]{fontenc}
\usepackage[utf8]{inputenc}
\usepackage[english]{babel}
\usepackage{graphicx} 
\usepackage{amsmath,amssymb}
\usepackage{bbm} 
\usepackage{hyperref}
\usepackage{epic}
\usepackage{tikz}
\usepackage{wrapfig}
\usepackage{caption}
\usepackage{dsfont}
\usepackage{ mathrsfs }
\usepackage{subcaption}

\usetheme{Madrid}

% Vire symbole de navigation
\setbeamertemplate{navigation symbols}{}
\setbeamercolor{block title}{use=structure,fg=structure.fg,bg=structure.fg!20!bg}
\setbeamercolor*{titlelike}{parent=palette primary}

%Titre = centrÈ + small caps + fond 
\setbeamertemplate{frametitle}[default][center]
\setbeamercolor{frametitle}{fg=RoyalPurple!70!Black,bg=structure.fg!5!bg}
\setbeamerfont{frametitle}{size=\normalsize,shape=\scshape,family=\rmfamily}

\setbeamercolor{CouleurBlock}{fg=structure.fg} 
\newcommand{\CB}[1]{{\usebeamercolor[fg]{CouleurBlock} #1}}

\setbeamercolor{Couleurexampleblock}{fg=block title Exemple.fg} 
\newcommand{\CEB}[1]{{\usebeamercolor[fg]{Couleurexampleblock} #1}}

\newcommand{\blanc}{\textcolor{white}{.}}

\def\Bcolorize<#1>{%
	\temporal<#1>{\color{Gray}}{\color{RoyalBlue}}{\color{Gray}}
}

%%%%%%%%%%%%%%%%%%%%%%%%%%%%%%%%%%%%%%%%%%%
%-------------- commandes ------------------

% -- mathds
\newcommand\1{{\ensuremath {\mathds 1} }} 

%----------------------------------------------------------------------------------------
%--- bb ----
\def\C{{\mathbb C}}
\def\CC{{\mathbb C}}
\def\FF{{\mathbb F}}
\def\HH{{\mathbb H}}
\def\bbI{{\mathbb I}}
\def\LL{{\mathbb L}}
\def\N{{\mathbb N}}
\def\NN{{\mathbb N}}
\def\Q{{\mathbb Q}}
\def\R{{\mathbb R}}
\def\RR{{\mathbb R}}
\def\UU{{\mathbb U}}
\def\Z{{\mathbb Z}}
\def\ZZ{{\mathbb Z}}




%----------------------------------------------------------------------------------------
%--- bold ----
\def\bA{{\bold A}}
\def\ba {{ \bold a}}
\def\bB{{\bold B}}
\def\bb{{\bold b}}
\def\bC{{\bold C}}
\def\bc{{\bold c}}
\def\be{{\bold e}}
\def\bI{{\bold I}}
\def\bj{{\bold j}}
\def\bk{{\bold k}}
\def\bL{{\bold L}}
\def\bl{{\bold \ell}}
\def\bm{{\bold m}}
\def\bp{{\bold p}}
\def\bQ{{\bold Q}}
\def\bq{{\bold q}}
\def\bR{{\bold R}}
\def\br{{\bold r}}
\def\bS{{\bold S}}
\def\bs{{\bold s}}
\def\bv{{\bold v}}
\def\bV{{\bold V}}
\def\bw{{\bold w}}
\def\bX{{\bold X}}
\def\bx{{\bold x}}
\def\by{{\bold y}}
\def\bz{{\bold z}}

\def\bone{{\bold 1}}
\def\btwo{{\bold 2}}
\def\bthree{{\bold 3}}
\def\bfour{{\bold 4}}
\def\bfive{{\bold 5}}
\def\bsix{{\bold 6}}
\def\bseven{{\bold 7}}

\def\bsigma{{\bold \sigma}}
\def\bnu{{\bold \nu}}
\def\bnull{{\bold 0}}

%----------------------------------------------------------------------------------------
%--- rm -----
\def\rc{{\mathrm{c}}}
\def\rd{{\mathrm{d}}}
\def\re{{\mathrm{e}}}
\def\ri{{\mathrm{i}}}
\def\rp{{\rm p}}
\def\rh{{\rm h}}

%--- operateurs ----
\def\Re{{\mathrm{Re }\, }}
\def\Im{{\mathrm{Im }\, }}

\def\tr{{\rm tr}}
\def\Tr{{\rm Tr}}
\def\VTr{ \underline{ \rm Tr \,}}
\def\div{{\rm div \,}}


%-------------------------------------------------------------------------------------------
%--- mathcal ---
\def\cA{{\mathcal A}}
\def\cB{{\mathcal B}}
\def\cC{{\mathcal C}}
\def\cD{{\mathcal D}}
\def\cE{{\mathcal E}}
\def\cF{{\mathcal F}}
\def\cG{{\mathcal G}}
\def\cH{{\mathcal H}}
\def\cI{{\mathcal I}}
\def\cJ{{\mathcal J}}
\def\cK{{\mathcal K}}
\def\cL{{\mathcal L}}
\def\cM{{\mathcal M}}
\def\cP{{\mathcal P}}
\def\cQ{{\mathcal Q}}
\def\cR{{\mathcal R}}
\def\cS{{\mathcal S}}
\def\cT{{\mathcal T}}
\def\cV{{\mathcal V}}
\def\cW{{\mathcal W}}
\def\cZ{{\mathcal Z}}


%-------------------------------------------------------------------------------------------
%---- mathfrak ------
\def\fs{{\mathfrak{S}}}
\def\fh{{\mathfrak{h}}}
\def\fS{{\mathfrak S}}
\def\fP{{\mathfrak P}}
\def\fH{{\mathfrak H}}

%-------------------------------------------------------------------------------------
% -- mathscr --
\newcommand{\sS}{\mathscr{S}}
\newcommand{\sC}{\mathscr{C}}
\newcommand{\sD}{\mathscr{D}}

%--------------------------------------------------------------------------------------
% -- misc --
\newcommand{\bra}{\langle}
\newcommand{\ket}{\rangle}

\def\spinup{{\uparrow}}
\def\spindown{{\downarrow}}

\newcommand{\cst}{\alpha} %for H = (- i \Nabla + \cst A)^2
\newcommand{\BZ}{{\Gamma^\star}} % Brillouin Zone
\newcommand{\WS}{\Gamma} %cellule Wigner-Seitz = \R^3 / \Lat
\newcommand{\RLat}{\mathcal{R}^*} % Reciprocal Lattice
\newcommand{\Lat}{\mathcal{R}} % Lattice
\newcommand{\rot}{\vec {\text{rot}\ }} % Rotationnel
\newcommand{\curl}{\textbf{curl}} %curl

\newcommand{\per}{\mathrm{per}}
\newcommand{\ess}{\mathrm{ess}}
\newcommand{\xc}{\mathrm{xc}}
\newcommand{\eff}{\mathrm{eff}}
\newcommand{\loc}{\mathrm{loc}}
\newcommand{\pv}{\mathrm{p.v.} }
\newcommand{\sgn}{\mathrm{sgn}}
\newcommand{\nuc}{\mathrm{nuc}}
\newcommand{\KS}{\mathrm{KS}}
\newcommand{\LSDA}{\mathrm{LSDA}}
\newcommand{\LDA}{\mathrm{LDA}}
\newcommand{\EF}{\mathrm{EF}}
\newcommand{\ext}{\mathrm{ext}}
\newcommand{\sym}{\mathrm{sym}}
\newcommand{\Supp}{\mathrm{Supp}}
\newcommand{\Hess}{\mathrm{Hess}}
\newcommand{\GW}{\mathrm{GW}}
\newcommand{\diag}{\mathrm{diag}}

\newcommand{\slater}{\mathrm{Slater}}
\newcommand{\mixed}{\mathrm{mixed}}
\newcommand{\pure}{\mathrm{pure}}


%---------- Math operator -------------------
\DeclareMathOperator*{\esssup}{ess\,sup}

%---------------------------------------------------
% For colors
\newcommand{\red}[1]{\textcolor{red}{#1}}
\newcommand{\blue}[1]{\textcolor{blue}{#1}}



% pour fins de preuves
\def\sqw{\hbox{\rlap{\leavevmode\raise.3ex\hbox{$\sqcap$}}$%
\sqcup$}}
\def\cqfd{\ifmmode\sqw\else{\ifhmode\unskip\fi\nobreak\hfil
\penalty50\hskip1em\null\nobreak\hfil\sqw
\parfillskip=0pt\finalhyphendemerits=0\endgraf}\fi}


%%%%%%%%%%%%%%%%%%%%%%%%%%%%%%%%%%%%%%%%%%%%%%
%%%%%%%%%%%%%%%%%%%%%%%%%%%%%%%%%%%%%%%%%%%%%%
%----- commentaire -----------
\newcommand{\commentaire}[1]{\textbf{*** #1  ***}}
\newcommand{\gab}[1]{\textbf{*** GABRIEL : #1  ***}}
\newcommand{\eric}[1]{\textbf{*** ERIC : #1  ***}}

% -- reference ----------------
\hypersetup{
    colorlinks,
    linkcolor={blue!50!blue},
    citecolor={red}
}
\renewcommand{\eqref}[1]{(\ref{#1})}

%-- modification----------------
\newcommand\addition[1]{\textcolor{blue}{#1}}
\newcommand\removal[1]{\textcolor{red}{\sout{#1}}}

















%%%%%%%%%%%%%%%%%%%%%%%%%%%%%%%%%%



%%%%%%%%%%%%%%%%%%%%%%%%%%%%%%%%%%%%%%%%%%%%%%%%%%%%%%%%%%%%%%%%%%%%

\title[$N$-representability]{Pure-state $N$-representability in current-spin-density functional theory}
\author[David Gontier]{David GONTIER\\
~\\
CERMICS, Ecole des Ponts ParisTech and INRIA}
\date{January 8, 2015}

\begin{document}

%%%%%%%%%%%%%%%%%%%%%%%%%%%%%%%%%%

\begin{frame} 
\titlepage
\end{frame} 


%%%%%%%%%%%%%%%%%%%%%%%%%%%%%%%%%%

\begin{frame}

\frametitle{The $N$-representability problem}

The set of admissible $N$-electron \blue{wave-functions} (wave functions with finite kinetic energy) is the fermionic space
\[
	\cW_N := \left\{ \Psi \in \bigwedge_{i=1}^N L^2(\R^3 \times \{ \spinup, \spindown\}, \C), \ \| \nabla \Psi \|_{L^2} < \infty, \  \| \Psi\|_{L^2} = 1 \right\},
\]
where $L^2(\R^3 \times \{ \spinup, \spindown\} ) = \left\{ \Phi = (\phi^\spinup, \phi^\spindown)^T, \  \| \Phi \|_{L^2}^2 := \int_{\R^3} | \phi^\spinup|^2 + | \phi^{\spindown}|^2 < \infty \right\}$. \\
~\\
For $\Psi \in \cW_N$, the \blue{electronic density} is
\[
	\rho_{\Psi}(\br) := N \sum_{(s_1, \cdots, s_N) \in \{ \spinup, \spindown\}^N} \int_{\R^{3(N-1)}} \left| \Psi (\br, s_1, \br_2, s_2, \cdots, \br_n, s_N) \right|^2 \rd^3 \br_2 \cdots \rd^3 \br_N.
\]
The \blue{$N$-representability problem} (for pure-states) is\\
\begin{center}
	\red{Is there a characterization of the set $I_N := \left\{ \rho_{\Psi}, \Psi \in \cW_N \right\}$?}
\end{center}


\end{frame}

%%%%%%%%%%%%%%%%%%%%%%%%%%%%%%%%%%%%%%%%%%%

\begin{frame}

\frametitle{Gilbert, Harriman, Lieb}

\begin{theorem} [Gilbert\footnotemark, Harriman\footnotemark, Lieb\footnotemark]
	\[
		\cI_N := \left\{ \rho \in L^1(\R^3), \quad \rho \ge 0, \quad \int_{\R^3} \rho = N, \quad \sqrt{\rho} \in H^1(\R^3) \right\}.
	\]
\end{theorem}

Together with the constrained search by Levy\footnotemark and Lieb\textsuperscript{3},
\begin{align*}
	E_N^0(v) := \inf_{\Psi \in \cW_N} \bra \Psi | H(v) | \psi \ket = \inf_{\rho \in \cI_N} \left\{ F_{LL}(\rho) + \int_{\R^3} v \rho \right\}.
\end{align*}
The first problem is linear, but very high-dimensional (\blue{curse of dimensionality}), the second problem is low-dimensional (but with an unknown functional).

\addtocounter{footnote}{-4}
 \stepcounter{footnote} \footnotetext{T.L. Gilbert. Phys. Rev. B, 502, 1975.}
 \stepcounter{footnote} \footnotetext{J.E. Harriman. Phys. Rev. A, 24,1981.}
 \stepcounter{footnote} \footnotetext{E.H. Lieb. Int. J. Quantum Chem., 24, 1983.}
  \stepcounter{footnote} \footnotetext{M. Levy, Proc. Natl. Acad. Sci. USA 76, 1979.}
  \addtocounter{footnote}{-4}

\end{frame}

%%%%%%%%%%%%%%%%%%%%%%%%%%%%%%%%%%%%%%

\begin{frame}

\frametitle{Outline}


\begin{center}
\textcolor{violet}{
	\huge{The $N$-representability problem with a magnetic field.}}
\end{center}

\end{frame}


%%%%%%%%%%%%%%%%%%%%%%%%%%%%%%%%%%%%%%

\begin{frame}

\frametitle{The Schrödinger-Pauli Hamiltonian}

We consider a system under a magnetic vector potential $\bA$. The \blue{Schrödinger-Pauli Hamiltonian} reads in atomic unit
\[
	H(v, \bA) := \sum_{k=1}^N \left( \dfrac12 (-\ri \nabla + \bA(\br_k))^2 + v(\br_k) \right) \mathds{I}_2 - \dfrac12 \sum_{k=1}^N \bB(\br_k) \cdot \bsigma_k + \sum_{1\le k < l \le N} \dfrac{1}{| \br_k - \br_l |} \mathds{I}_2,
\]
and it holds
\[
	\bra \Psi | H(v, \bA) | \Psi \ket = \bra \Psi | H(0, \bnull) | \Psi \ket + 
		\int_{\R^3} \tr_{\C_2} \left[ U(v, \bA) R_\Psi \right] + \int_{\R^3} \bA \cdot \bj_\Psi,
\]
where
\[
	U(v, \bA) :=\dfrac12 \begin{pmatrix} 2v - B_z + | \bA |^2 & -  B_x + \ri B_y \\ -  B_x - \ri  B_y & 2v +  B_z +  | \bA |^2 \end{pmatrix}.
\]
Here, $\bB = \curl(\bA)$ is the \blue{magnetic field}. 
\end{frame}


%%%%%%%%%%%%%%%%%%%%%%%%%%%%%%%%%%%%%%

\begin{frame}

\frametitle{The spin-polarized density matrix and the paramagnetic current}

We introduced the following $\Psi$-dependent quantities: \\
~\\
$\bullet$ The \blue{spin-polarized density $2 \times 2$ matrix $R_\Psi$}, defined by
\[
	R_\Psi = \begin{pmatrix} \rho_\Psi^{\spinup \spinup} & \rho_\Psi^{\spinup \spindown} \\ \rho_\Psi^{\spindown \spinup} & \rho_\Psi^{\spindown \spindown} \end{pmatrix}
\]
where, for $\alpha, \beta \in \{ \spinup, \spindown\}^2$,
\[
	\rho^{\alpha \beta}_{\Psi}(\br) := N \sum_{\vec{s} \in \{ \spinup, \spindown\}^{N-1}} \int_{\R^{3(N-1)}}\overline{\Psi (\br, \alpha,  \vec{\bz}, \vec{s})} \Psi (\br, \beta, \vec{\bz}, \vec{s})\  \rd^{3(N-1)} \vec{\bz}.
\]
Note that $R_\Psi$ is an hermitian function-valued $2 \times 2$ matrix, and it holds 
\[
	\tr_{\C^2} [ U(v,\bA) R_\Psi ] =  \left( v + \frac{| \bA |^2}{2} \right) \rho -\dfrac12  \bB \cdot \bm,
\] 
where $\rho$ is the \blue{total electronic density}, and $\bm$ is the \blue{spin angular momentum density}. \\
\red{Remark:} $R$ contains exactly the same information as the pair $(\rho, \bm)$.\\
~\\
$\bullet$ The \blue{paramagnetic current $\bj_\Psi$} is the vector-valued function
\[
	\bj (\br) = \Im \left( N \sum_{\vec{s} \in \{ \spinup, \spindown\}^{(N)}} \int_{\R^{3(N-1)}}  \overline{\Psi(\br, \vec{\bz}, \vec{s})}\  \nabla_{\br'} \Psi(\br', \vec{\bz}, \vec{s}) \Big|_{\br' = \br} \rd^{3(N-1)} \vec{\bz} \right).
\]
%Recall that the physical current is $\bj_{\textrm{phys}} := \bj + \rho \bA$.
\end{frame}



%%%%%%%%%%%%%%%%%%%%%%%%%%%%%%%%%%%%%%

\begin{frame}

\frametitle{Constrained search and representability}

We apply the constrained-search, and obtain
\[
	E_N^0 (v, \bA) = \inf_{\Psi \in \cW_N} \bra \Psi | H(v, \bA) | \Psi = \inf_{(R, \bj) \in \cK_N} \left\{ F(R, \bj) + \int_{\R^3} \tr_{\C^2} \left[ U(v, \bA) R \right] + \bA \cdot \bj \right\},
\]
which leads to the \blue{current-spin-density functional theory} (CSDFT).\\
~\\
 The $N$-representability problem (for pure-state) is
\begin{center}
	\red{Is there a characterization of the CSDFT-set $\cK_N := \left\{ (R_\Psi, \bj_\Psi), \ \Psi \in \cW_N \right\}$?}
\end{center}

\red{Remark:} 
\begin{itemize}
	\item If we neglect the spin-effects (\textit{i.e.} the Zeeman term $\bB \cdot \bm$), this energy depends only on the pair $(\rho, \bj)$. This leads to the \blue{current-DFT} (CDFT).
	\item If we neglect the orbital term (\textit{i.e.} $\bA \cdot \bj$), this energy depends only on $R$. This leads to the \blue{non-collinear spin-DFT} (SDFT).
\end{itemize}

\begin{center}
	\red{$\Longrightarrow$ Is there a characterization of the SDFT-set $\cJ_N := \left\{ R_\Psi, \ \Psi \in \cW_N \right\}$?}
\end{center}


\end{frame}

%%%%%%%%%%%%%%%%%%%%%%%%%%%%%%%%%%%%%%

\begin{frame}

\frametitle{Slater determinants and mixed-state}

\red{Slater determinants}:\\
A special case of wave functions are the \blue{Slater determinants} of $N$ orthonormal functions $\left\{ \Phi_k = (\phi_k^\spinup, \phi_k^\spindown)^T \right\}_{1 \le k \le N}$:
\[
	\sS[\Phi_1, \cdots, \Phi_N] := \dfrac{1}{\sqrt{N!}} \det \left( \Phi_i (\br_j) \right)_{1 \le i,j \le N}.
\]
For such a wave-function, the previous quantities have a very special form, namely
\[
	R = \sum_{k=1}^N \begin{pmatrix} | \phi^\spinup_k |^2 & \phi^\spinup_k \overline{\phi^\spindown_k} \\ \overline{\phi^\spinup_k} \phi^\spindown_k & | \phi^\spindown_k |^2 \end{pmatrix}, 
	\quad
	\rho = \sum_{k=1}^N  | \phi^\spinup_k |^2 +  | \phi^\spindown_k |^2
	\quad \text{and} \quad
	\bj = \sum_{k=1}^N \Im \left( \overline{\phi_k^\spinup} \nabla \phi_k^\spinup + \overline{\phi_k^\spindown} \nabla \phi_k^\spindown \right)
\]

\red{Mixed-states}:\\
It is possible to extend the notion of $R$, $\rho$ and $\bj$ to \blue{mixed-states}. We will say that a spin-density matrix $R$ is \blue{mixed-state representable} (and similarly for $\rho$ and $\bj$) if $R$ can be written as a convex combination of pure-state representable spin-density matrices:
\[
	R = \sum_{k=1}^\infty \lambda_k R_k, \quad 0 \le \lambda_k \le 1, \quad \sum_{k=1}^\infty \lambda_k = 1, \quad \text{and} \quad R_k \quad \text{pure-state representable}.
\]



\end{frame}


%%%%%%%%%%%%%%%%%%%%%%%%%%%%%%%%%%%%%%

\begin{frame}

\frametitle{Outline}


\begin{center}
\textcolor{violet}{
	\huge{Results (previous and new).}}
\end{center}

\end{frame}

%%%%%%%%%%%%%%%%%%%%%%%%%%%%%%%%%%%%%%


\begin{frame}


\frametitle{$N$-Representability for SDFT}

\red{Representability for SDFT} (in the pure-state case, and in the mixed-state case):

\begin{theorem}[DG 2013\footnotemark]
	$\bullet$ A spin-density matrix $R$ is mixed-state representable iff
	\[
		R \in \cJ_N :=  \left\{ R \in \cM_{2 \times 2} (L^1(\R^3)), \ R \ge 0, \ \int_{\R^3} \tr_{\C^2} \left[ R \right] = N, \ \sqrt{R} \in \cM_{2 \times 2} (H^1(\R^3, \C)) \right\}.
	\]
	If $N \ge 2$, it is also pure-state representable (by a Slater determinant).\\
	~\\
	$\bullet$ \textbf{Case $N=1$}\\
	 A spin-density matrix $R$ is (pure-state) representable by a single orbital iff
	\[
		R \in \cJ_1 \quad \text{and} \quad \det{R} \equiv 0.
	\]
\end{theorem}

\begin{itemize}
	\item The $\sqrt{}$ is in the hermitian matrices sense
	\item  It is a natural extension of the previous result for $\rho$:
\end{itemize}
	\[
		\cI_N = \left\{ \rho \in L^1(\R^3), \quad \rho \ge 0, \quad \int_{\R^3} \rho = N, \quad \sqrt{\rho} \in H^1(\R^3) \right\}.
	\]

\footnotetext{D. Gontier. Phys. Rev. Lett. 111, 2013.}
\addtocounter{footnote}{-1}
\end{frame}

%%%%%%%%%%%%%%%%%%%%%%%%%%%%%%%%%%%%%%%


\begin{frame}

\frametitle{Application to LSDA}

In the \blue{Local Spin-Density Approximation} (LSDA) introduced by von Barth and Hedin\footnotemark, we write
\begin{equation} \label{vonBarth}
	E_{\xc}(R) \approx E_{\xc}^{\LSDA}(\rho^+, \rho^-) := \dfrac12 \left[ E_{\xc}^{\LDA} (2 \rho^+) + E^{\LDA}_{\xc} (2 \rho^-) \right]
\end{equation}
where $\rho^{+/-}$ are the two \blue{eigenvalues} of the $2 \times 2$ matrix $R$, and $E_{\xc}^{\LDA}$ is the standard \blue{exchange-correlation functional} in the non-polarized case.\\


\begin{lemma}[DG 2013]
	If $R \in \cJ_N$, then its eigenvalues $\rho^{+,-}$ satisfy
	\[
		\rho^{+,-} \in L^1(\R^3),  \quad \rho^{+,-} \ge 0, \quad \sqrt{\rho^{+/-}} \in H^1(\R^3).
	\]
	(the converse is false).
\end{lemma}
~\\
\red{Remark}: $E_{\xc}^{\LSDA}$ is well-defined if $E_{\xc}^{\LDA}$ is well-defined.



\footnotetext{U. von Barth and L. Hedin. J. Phys. C 5,  1972.}
\addtocounter{footnote}{-1}

\end{frame}



%%%%%%%%%%%%%%%%%%%%%%%%%%%%%%%%%%%%%%%

\begin{frame}

\frametitle{Previous Work in CDFT}

\red{Previous results for representability in CDFT} %(in the pure-state case, and in the mixed-state case).

\begin{theorem}[Tellgren, Kvall, Helgaker, 2014\footnotemark]
	The pair $(\rho, \bj)$ is mixed-state representable whenever
	\[
		\rho \in \cI_N, \quad \dfrac{| \bj |^2}{\rho} \in L^1(\R^3)
		\quad \text{and} \quad
		(1 + | \cdot |^2) \rho | \nabla (\rho^{-1}  \bj) |^2 \in L^1(\R^3)
		\quad \text{(mild condition).}
	\]
\end{theorem}

%\vspace{-0.5em}

\begin{theorem}[Lieb, Schrader, 2013\footnotemark]
	The pair $(\rho, \bj)$ is pure-state representable (by a Slater determinant) whenever
	\[
		\rho \in \cI_N, \quad \dfrac{| \bj |^2}{\rho} \in L^1(\R^3) 
		\quad \text{with} \quad
		 \text{$N \ge 4$ and other mild conditions\footnotemark}.
	\]
\end{theorem}

\red{Remark}: The conditions $\rho \in \cI_N$ and $| \bj |^2 / \rho \in L^1(\R^3)$ are also necessary conditions.

\vspace{2em}


\addtocounter{footnote}{-2}
\footnotetext{E.I. Tellgren, S. Kvaal, and T. Helgaker. Phys. Rev. A, 89, 2014.}
\addtocounter{footnote}{1}
\footnotetext{E.H. Lieb and R. Schrader. Phys. Rev. A, 88, 2013.}
\addtocounter{footnote}{1}
\footnotetext{
Namely, if $\bw = \curl( \bj / \rho)$,
\begin{align*}
		\sup_{\br \in \R^3} & ( 1 + (r_1)^2)^{(1+\delta)/2}  ( 1 + (r_2)^2)^{(1+\delta)/2}  ( 1 + (r_3)^2)^{(1+\delta)/2} \left( | \bw (\br) | + | \nabla \bw(\br) | \right)< \infty \\
		%\sup_{\br \in \R^3} & ( 1 + (r_1)^2)^{(1+\delta)/2}  ( 1 + (r_2)^2)^{(1+\delta)/2}  ( 1 + (r_3)^2)^{(1+\delta)/2} | \nabla \bw (\br) | < \infty,
\end{align*}}
\addtocounter{footnote}{-3}

\end{frame}





%%%%%%%%%%%%%%%%%%%%%%%%%%%%%%%%%%%%%%%%%

\begin{frame}

\frametitle{Representability for CSDFT, case $N =1$}

\red{Representability for CSDFT, for $N =1$}

\begin{lemma}[DG]
	A necessary condition for a pair $(R, \bj)$ to be pure-state representable by a single orbital (having smooth enough global phases) is\footnotemark
	\[
		R \in \cJ_1 \quad \text{with} \quad \det(R) \equiv 0
		\quad \text{and} \quad
		\curl \left( \dfrac{\bj}{\rho} - \dfrac{\Im( \overline{\rho^{\spinup \spindown}} {\nabla \rho^{\spinup \spindown}})}{\rho \rho^{\spindown \spindown}} \right) = \bnull
	\]
\end{lemma}
~\\
\red{Remark}: We recover the traditional \blue{curl-free condition} in the unpolarized case, which amounts to setting $\rho^{\spinup \spindown} = 0$: a necessary condition for a pair $(\rho, \bj)$ to be pure-state representable by a single orbital (having a smooth enough global phase) is 
\[
	 \rho \in \cI_1
	 \quad \text{and} \quad
	 \curl \left( \dfrac{\bj}{\rho} \right) = \bnull .
\]

\footnotetext{Recall that
\[
R = \begin{pmatrix}
	\rho^{\spinup \spinup} & \rho^{\spinup \spindown} \\ \rho^{\spindown \spinup} & \rho^{\spindown \spindown}
\end{pmatrix}
\]
}
\addtocounter{footnote}{-1}


\end{frame}



%%%%%%%%%%%%%%%%%%%%%%%%%%%%%%%%%%%%%%%%%

\begin{frame}

\frametitle{Representability for CSDFT, case $N \ge 12$}

\red{Representability for CSDFT, for $N \ge 12$}

\begin{theorem}[DG]
	A pair $(R, \bj)$ is pure-state representable (by a Slater determinant) whenever
		\[
		R \in \cJ_N, \quad
		\dfrac{| \bj |^2}{\rho} \in L^1(\R^3), 
		\quad \text{with} \quad
		\text{$N \ge 12$ and the same previous mild conditions}
	\]
\end{theorem}

\red{Remark 1}: To prove the result, we decompose $R$ into $3$ well-behaved matrices $R = R_1 + R_2 + R_3$, and we apply on each $R_k$ the Lieb and Schrader result (which holds for $N \ge 4$). Hence the result for $N \ge 3 \times 4 = 12$. \\
~\\
\red{Remark 2}: We believe that the result also holds for some $N < 12$ but we do not have a proof of this fact. %From the Lieb and Schrader's paper, we know that the result is false for $N = 2$.

\begin{corollary}
	A pair $(R, \bj)$ is mixed-state representable whenever
	\[
		R \in \cJ_N, \quad
		\dfrac{| \bj |^2}{\rho} \in L^1(\R^3), 
		\quad \text{with} \quad
		\text{$N \in \N^*$ and the same previous mild conditions}
	\]
\end{corollary}

\end{frame}

%%%%%%%%%%%%%%%%%%%%%%%%%%%%%%%%%%%%%%%%%

\begin{frame}

\frametitle{Final remarks}

\red{Summary}: \\
 $\bullet$ We gave necessary and sufficient conditions for pure-state $N$-representability in SDFT.\\
 $\bullet$ We gave sufficient conditions for pure-state $N$-representability in CSDFT when $N \ge 12$.\\
 $\bullet$ When $N =1$, there is a non-trivial interplay between the spin-density $R$ and the paramagnetic current $\bj$, namely
 \[
 	\curl \left( \dfrac{\bj}{\rho} - \dfrac{\Im( \overline{\rho^{\spinup \spindown}} {\nabla \rho^{\spinup \spindown}})}{\rho \rho^{\spindown \spindown}} \right) = \bnull.
 \]

\red{Comments and future work}:

$\bullet$ Our results use the so-called \blue{Lazarev-Lieb orthogonalization process}\footnotemark. In particular, we were not able to bound the \blue{kinetic energy} of the representing Slater determinants. \\
$\bullet$ We leave the question $N < 12$ open.

~\\
\begin{center}
\red{Thank you for your attention!}
\end{center}

\footnotetext{E.H. Lieb and O. Lazarev. Indiana Univ. Math. Jour., 2014.}



\end{frame}


%%%%%%%%%%%%%%%%%%%%%%%%%%%%%%%%%%%%%%%%%

\end{document}