\documentclass[10pt]{beamer}
\newcommand{\R}{\mathbb{R}}
\newcommand{\C}{\mathbb{C}}
\newcommand{\Z}{\mathbb{Z}}
\newcommand{\N}{\mathbb{N}}
\DeclareMathOperator*{\argmin}{arg\,min}
\DeclareMathOperator{\cotan}{cotan}
\DeclareMathOperator{\sinc}{sinc}
\DeclareMathOperator{\Tr}{Tr}
\DeclareMathOperator{\tr}{tr}
\DeclareMathOperator{\Gram}{Gram}
\DeclareMathOperator{\diag}{diag}
\newcommand{\lp}{\left(}
  \newcommand{\rp}{\right)}
\usepackage{blkarray}
\usepackage{multirow}
\usepackage{tikz}
% \usepackage{algorithm2e}
% \usepackage{algorithmic}
\usepackage{float}
\usepackage{framed}
% \usepackage{ulem}
\usetikzlibrary{shapes,arrows}

\newcommand{\lela}{\left \langle}
  \newcommand{\rira}{\right \rangle}
\newcommand{\norm}[1]{\left\lVert#1\right\rVert}
\newcommand{\abs}[1]{\left|#1\right|}

\usepackage{tikz}
\usetikzlibrary{shapes,arrows}
\usepackage[french,english]{babel}
\usepackage[utf8]{inputenc}
\usepackage[T1]{fontenc}
\usepackage{lmodern}
% \usepackage{common}
\usepackage{amsmath,amsfonts,amssymb}
% \usepackage{overpic,boxedminipage}
\usepackage{hyperref}
\usetheme{Warsaw}
\setbeamercolor*{block body alerted}{bg= blue!10}
\setbeamercolor*{block title alerted}{bg= blue!50}
% \usefonttheme{professionalfonts}
\setbeamertemplate{navigation symbols}{}
\setbeamertemplate{headline}{}
\setbeamertemplate{footline}{}
% \setbeamertemplate{blocks}[rounded][shadow=false]
\title[iX_Blue]{SEME 2015: iX-Blue Optimisation de trajectoire pour la navigation Bathymetrique}
\author{Kieran Delamotte, Carlo de Franchis, David Gontier, Antoine Levitt, Fraçois Madiot, Carlo Marcati}
\date{SEME, Vendredi 16 Janvier 2015}
\institute{Sujet proposé par Jérémy Nicola, iXBlue}
% \institute{CEA, DAM, DIF\\Collaboration avec Marc Torrent}
\newcommand{\loja}{\L{}ojasiewicz\xspace}
\begin{document}

\frame{\titlepage}

\frame{\tableofcontents}
\AtBeginSection[]{
  \begin{frame}{Summary}
    \tableofcontents[currentsection, hideothersubsections]
  \end{frame}
}

\section{Présentation du problème}
\subsection{Contexte}
\frame{
  \frametitle{iXBlue}
  \begin{itemize}
  \item Activité principale : conception de centrales inertielles
  \item Interféromètre à base de fibre optique : donne les
    accélérations selon les six degrés de liberté
  \item Information locale soumise à dérive. Nécessité d'un recalage
    par des informations globales
  \item Sous-marin : pas possible d'utiliser un GPS
  \item Idée : mesurer le fond marin, et comparer avec des cartes bathymétriques
  \end{itemize}
}
\subsection{Incertitudes et ordres de grandeur}
\section{Modélisation et algorithmes de recalage}
\subsection{Modélisation des incertitudes}
\subsection{Approche par intervalles}
\subsection{Algorithme de recalage par boites}
Schéma avec trois points

Example avec les trucs blancs
\section{Optimisation de trajectoire}
\subsection{Gloutonne}
\subsection{Dijkstra}
\subsection{Continu}
\section{Future research}
Matrices

Probas ? Bayésien ?

Feature detection ?
\section{Conclusion}

\subsection{Context}
\frame{
  \frametitle{Physical context}
  \begin{itemize}
  \item Matter at the atomic level is described by quantum mechanics
  \item Main equation : time-independent Schrödinger equation $$H \psi
    = E \psi$$ for the electronic wavefunction $\psi$
  \item Eigenvalue equation for the self-adjoint operator $H$ on a
    Hilbert space $\mathcal H$
  \item For $N$ electrons, $\psi \in \mathcal H$ is a function of $3N$
    variables
  \end{itemize}
}
\end{document}
