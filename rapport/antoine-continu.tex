\paragraph{Modèle jouet}
Les modèles précédents sont discrets par nature. Il est nécessaire de
se poser la question d'une modélisation continue (en espace et en
temps), soit qu'on espère que ces modèles sont les modèles limites des
précédents quand les échelles d'espace et de temps sont petites, soit
réciproquement pour obtenir de nouvelles idées d'algorithmes basés sur
une modélisation continue.

Dans cette partie, on suppose que la bathymétrie $h(x,y)$ est continue
et suffisamment régulière. L'inconnue est le chemin $\gamma : [0,1]
\to \R^{2}$, également supposé régulier.

Une première idée est de maximiser la pente parcourue, avec une
fonctionelle de type
  \begin{align*}
    c(\gamma) = -\int_{t=0}^{1} |\nabla h(\gamma(t))|^{2}  dt
  \end{align*}

  Ce problème est très mal posé: l'infimum n'est pas atteint, et les
  suites minimisantes se localisent sur les maximas de $\nabla h$, ce
  qui implique une perte de régularité de $\gamma$. Il
  est donc naturel de considérer une régularisation de ce problème
  pour éviter ce type de comportement pathologique, en pénalisant les
  $\gamma$ trop irréguliers :
\begin{align*}
  c(\gamma) =  \int_{t=0}^{1} a |\gamma'(t)|^{2} - b |\nabla
  h(\gamma(t))|^{2} dt
\end{align*}

Ce modèle nous semble être bien posé. Son étude mathématique et
numérique pourrait être intéressante, mais nous n'avons pas eu le
temps de la considérer. Cependant, il ne prend pas en compte
l'intuition (supportée par les simulations) que $\gamma$ doit alterner
les directions de $\nabla h$ pour localiser le sous-marin. Ce type
d'information nécessite une prise en compte de l'historique de
$\gamma$, et donc une fonction de coût non-locale, dont l'analyse est
plus compliquée.
% \begin{align*}
%   c(\gamma) =  \int_{t=0}^{1} a |\gamma'(t)|^{2} - b |\nabla
%   h(\gamma(t)) \cdot \gamma'(t)| dt
% \end{align*}

\paragraph{Un modèle un peu plus complexe}
Une autre possibilité est de modéliser l'évolution des incertitudes au
long de la trajectoire. Par exemple, en modélisant séparément les
incertitudes en $x$ et en $y$, on pourrait écrire
\begin{align*}
  c(\gamma) &= \sigma_{x}(1) + \sigma_{y}(1), \text{ où}\\
  \dot \sigma_{x} &= a - b \sigma_{x} (\partial_{x} h(\gamma(t)))^{2}\\
  \dot \sigma_{y} &= a - b \sigma_{y} (\partial_{y} h(\gamma(t)))^{2}.
\end{align*}

Ce modèle présente un amortissement exponentiel de $\sigma_{x}$ vers
un point d'équilibre
$\sigma_{x} \sim 1/(\partial_{x} h(\gamma(t)))^{2}$. Il est sous la
forme générale d'un contrôle optimal, et pourrait donc être traité par
ces méthodes (dont nous ne sommes pas spécialistes).

Un défaut de cette modélisation est que le modèle n'est pas isotrope
(il ne traite pas de la même façon les trajectoires dans des
directions différentes). Une généralisation possible pourrait être de
représenter l'incertitude par une matrice SDP $A$, dont les directions
et valeurs propres représenteraient une ellipse d'incertitude. Si on
pose $G = \nabla h \nabla h^{T}$, on peut considérer le modèle
\begin{align*}
  \dot A = a A - b \sqrt G A \sqrt G,
\end{align*}
analogue du précédent, mais traitant de la même façon toutes les
directions. On réduit ici les incertitudes dans la direction
$\nabla h$, et on ne les modifie pas dans la direction
$\nabla h^{\perp}$. Ce modèle est toutefois incohérent car $A$ ne
reste pas forcément SDP. Nous ne savons pas si il y a une ``bonne''
généralisation isotrope du modèle précédent.
