\documentclass[11pt,oneside]{article}
\newcommand{\R}{\mathbb{R}}
\newcommand{\C}{\mathbb{C}}
\newcommand{\Z}{\mathbb{Z}}
\newcommand{\N}{\mathbb{N}}
\usepackage[french]{babel}
\usepackage[utf8]{inputenc}
\usepackage[T1]{fontenc}
\usepackage{lmodern}
\usepackage{array}
\usepackage{multicol}
\usepackage{titlesec}
\usepackage{graphicx}
\usepackage[margin=3cm]{geometry}
% \usepackage{fullpage}
% \usepackage{common}
\usepackage{amsmath,amsfonts,amssymb}
% \usepackage{overpic,boxedminipage}
\usepackage{hyperref}
\title{iXBlue}
\author{}
\date{}
\begin{document}
\maketitle
% \setcounter{secnumdepth}{1}
\setcounter{tocdepth}{1}
\tableofcontents

\section{Introduction}
\section{Recalage}
\section{Optimisation de trajectoire}
\section{Pistes continues}
\section{Pistes probabilistes}
\section{Conclusion}
\section{Bibliographie}
\end{document}
