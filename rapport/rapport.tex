\documentclass[11pt,oneside]{article}
\newcommand{\R}{\mathbb{R}}
\newcommand{\C}{\mathbb{C}}
\newcommand{\Z}{\mathbb{Z}}
\newcommand{\N}{\mathbb{N}}
\usepackage[french]{babel}
\usepackage[utf8]{inputenc}
\usepackage[T1]{fontenc}
\usepackage{lmodern}
\usepackage{array}
\usepackage{multicol}
\usepackage{titlesec}
\usepackage{graphicx}
\usepackage[margin=3cm]{geometry}
% \usepackage{fullpage}
% \usepackage{common}
\usepackage{amsmath,amsfonts,amssymb}
% \usepackage{overpic,boxedminipage}
\usepackage{hyperref}
\title{iXBlue}
\author{}
\date{}
\begin{document}
\maketitle
% \setcounter{secnumdepth}{1}
\setcounter{tocdepth}{1}
\tableofcontents

\section{Introduction}
\section{Recalage}
\section{Optimisation de trajectoire}
\section{Pistes continues}
\paragraph{Modèle jouet}
  On cherche à aller sur des zones de grande pente:
  \begin{align*}
    c(\gamma) = -\int_{t=0}^{1} |\nabla h(\gamma(t))|^{2}  dt
  \end{align*}

  Mal posé (les suites minimisantes se localisent sur les maxima de
  $\nabla h$) : régularisation~?
  
\begin{align*}
  c(\gamma) =  \int_{t=0}^{1} a |\gamma'(t)|^{2} - b |\nabla
  h(\gamma(t))|^{2} dt
\end{align*}

Bien posé~? Minimiseurs~? Comment prendre en compte l'heuristique que
$\gamma$ doit alterner les directions de $\nabla h$ pour se localiser~?

% \begin{align*}
%   c(\gamma) =  \int_{t=0}^{1} a |\gamma'(t)|^{2} - b |\nabla
%   h(\gamma(t)) \cdot \gamma'(t)| dt
% \end{align*}

\paragraph{Un modèle un peu plus complexe}
Modèle effectif pour les incertitudes en $x$ et $y$ $\sigma_{x}$ et
$\sigma_{y}$ :
\begin{align*}
  c(\gamma) &= \sigma_{x}(1) + \sigma_{y}(1), \text{ où}\\
  \dot \sigma_{x} &= a - b \sigma_{x} (\partial_{x} h(\gamma(t)))^{2}\\
  \dot \sigma_{y} &= a - b \sigma_{y} (\partial_{y} h(\gamma(t)))^{2}.
\end{align*}
Amortissement exponentiel vers un point d'équilibre $\sigma_{x} \sim
1/(\partial_{y} h(\gamma(t)))^{2}$. Sous la forme générale d'un
contrôle optimal. Problème : modèle non isotrope (trajectoires
diagonales~?)

Généralisation : représenter l'incertitude par une matrice SDP $A$. Si
on pose $G = \nabla h \nabla h^{T}$,
\begin{align*}
  \dot A = a A - b \sqrt G A \sqrt G.
\end{align*}
On modifie les incertitudes dans la direction $\nabla h$, on n'y
touche pas dans $\nabla h^{\perp}$. Problème : $A$ ne reste
pas forcément SDP. Y a-t-il une bonne généralisation~?

\section{Pistes probabilistes}
  \paragraph{Modélisation des incertitudes}
\begin{itemize}
\item Sources d'incertitude : précision de l'accéléromètre, erreurs
  d'intégration (fréquence finie), erreur numérique (stockage en
  virgule flottante)
\item Dérive de l'estimation de position
\item Modélisation délicate
\item Si on fait une erreur initiale sur l'accélération, dérive en
  $t^{2}$
\item Si on fait une erreur initiale sur la vitesse, dérive en
  $t$
\item Compensation des erreurs~? Marche aléatoire, dérive en $\sqrt t$
 ~? $t^{3/2}$~? $t^{5/2}$~?
\item Non trivial en pratique
\end{itemize}
\paragraph{Modélisation probabiliste}
  \begin{itemize}
  \item Une approche par intervalles néglige la compensation des erreurs
    et est trop pessimiste. Modèle probabiliste~?
  \item Soit $(x_{0}, x_{1} \dots, x_{N})$ le chemin suivi par le
    sous-marin, supposé exact.
  \item Soit $X_{n}$ la position estimée du sous-marin. Loi de
    $X_{n+1}$~?
  \item Sans bathymétrie, on met à jour la position avec le vecteur
    vitesse $v_{n} = x_{n+1} - x_{n}$, entaché (pour simplifier) d'une erreur aléatoire
    $\varepsilon_{v} \sim N(0, \sigma_{v})$
  \item La bathymétrie nous donne $h(X_{n+1}) = h(x_{n+1}) +
    \varepsilon_{h}$, avec une erreur $\varepsilon_{h} \sim N(0, \sigma_{h})$
  \end{itemize}

  \begin{center}
  \boxed{\mbox{$X_{n+1} \sim L(X_{n} + v_{n} + \varepsilon_{v} \;\Big|\;
    h(X_{n} + v_{n} + \varepsilon_{v}) = h(x_{n+1}) + \varepsilon_{h})$}}
  \end{center}
  \begin{center}
  \boxed{\mbox{$X_{n+1} \sim L(X_{n} + v_{n} + \varepsilon_{v} \;\Big|\;
    h(X_{n} + v_{n} + \varepsilon_{v}) = h(x_{n+1}) +  \varepsilon_{h})$}}
  \end{center}

  En notant formellement $P(X=x)$ pour la densité de $X$ en $x$,
  \begin{align*}
    P(X_{n+1} = x) &= P(X_{n} + v_{n} + \varepsilon_{v} = x \;\Big|\;
    h(X_{n} + v_{n} + \varepsilon_{v}) = h(x_{n+1}) + 
                     \varepsilon_{h})\\
    &= \frac 1 N P(X_{n} + v_{n} + \varepsilon_{v} = x \cap
    h(X_{n} + v_{n} + \varepsilon_{v}) = h(x_{n+1}) + \varepsilon_{h})\\
    &= \frac 1 N P(X_{n} + v_{n} + \varepsilon_{v} = x) \; P(h(x) = h(x_{n+1}) + \varepsilon_{h}),
  \end{align*}
  où $N$ est choisi pour normaliser $X_{n+1}$.

  On calcule $P(h(x) = h(x_{n+1}) + \varepsilon_{h})$ (facile si
  $\varepsilon_{h}$ est gaussienne), on calcule $P(X_{n} + v_{n} +
  \varepsilon_{v} = x)$ (translation et convolution), on multiplie et on
  normalise. En théorie bien plus précis que l'approche par boites,
  mais quel traitement des incertitudes de vitesse~? Lien avec les
  filtres de Kalman~?
\section{Conclusion}
\section{Bibliographie}
\end{document}
